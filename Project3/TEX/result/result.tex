\subsection{SIRS modelling}
	Here we confirmed that the recovery rate, $b$, should be smaller than the rate of transmission, $a$, to sustain a disease. We saw that there is a surprisingly good match between the RK4 and the MC model, but the MC model needs more time, approximately 3 years,  to stablize and find a steady - state in the population. Finally we figured that the MC simulation crashes if the number of infected people reaches zero. Details can be read below.
	
	The equilibrium expectation values and corresponding standard deviation, after 10 years of time or 3650 days, were found to be: \autocite{Nobody06}

\begin{table}[h]
\center
\begin{tabular}{llll}
      & S(t)   & I(t)   & R(t)   \\
Mean  & 108.81 & 110.11 & 181.06 \\
Stdev & 39.48  & 32.06  & 28.24  \\
\end{tabular}
\end{table}

	 
\subsubsection{4th order Runge-Kutta on the ODEs from SIRS, testing recovery rate.}
The simplest form of SIRS models differential equations solved with 4th order Runge - Kutta, then plottet. All plots in figure: \ref{fig:4RKSIRS}, have a black line that represent the change in total population. One can see there is not much change in population, as expected, at this moment. The equilibrium-states, discussed in section: \ref{sec:stady_state}, are plottet with stapled lines. These solutions are only plottet for one year, due to their obvious approach of the equilibrium-states. The difference between these figures are due to different recovery rate. One can see form these figures that after the initial burst of infection most develops an immunity after a short time and the disease reaches a steady-state. We can also confirm that if the transmission rate is larger then the recovery rate ($b<a$) the disease mange to get some kind of foothold, to sustain itself in the population.

\begin{figure}[H]
    \centering
    \begin{subfigure}{0.49\textwidth}
        \centering
        \includegraphics[width=\linewidth]{../fig/texfig/RK4_b1T1.png}
        \caption{Recovery rate, $b = 1$}
    \end{subfigure}%
     ~ 
    \begin{subfigure}{0.49\textwidth}
         \centering
         \includegraphics[width=\linewidth]{../fig/texfig/RK4_b2T1.png}
         \caption{Recovery rate, $b = 2$}
    \end{subfigure}
     ~ 
    \begin{subfigure}{0.49\textwidth}
         \centering
         \includegraphics[width=\linewidth]{../fig/texfig/RK4_b3T1.png}
         \caption{Recovery rate, $b = 3$}
    \end{subfigure}
     ~ 
    \begin{subfigure}{0.49\textwidth}
         \centering
         \includegraphics[width=\linewidth]{../fig/texfig/RK4_b4T1.png}
         \caption{Recovery rate, $b = 4$}
    \end{subfigure}
    \caption{4th order Runge-Kutta on ODEs given for SIRS model, where the only difference is the recovery rate,b, that is equal to $1,2,3,4$ respectivly}
    \label{fig:4RKSIRS}
\end{figure}

\subsubsection{Neural networks}

	 \begin{figure}[H]
	     \centering
	     \begin{subfigure}{0.5\textwidth}
	         \centering
	         \includegraphics[width=\linewidth]{result/Resultater_supervised/tanh_1000_100_001.pdf}
	         \caption{0.001 Step size limit for tanh activation function}
	     \end{subfigure}%
	     ~ 
	     \begin{subfigure}{0.5\textwidth}
	         \centering
	         \includegraphics[width=\linewidth]{result/Resultater_supervised/tanh_1000_100_01.pdf}
	         \caption{0.01 Step size limit for tanh activation function}
	     \end{subfigure}
	     \caption{Neural network results compared to the RK4 solution for two different step sizes}
	     \label{fig:step size}
	 \end{figure}
 	When comparing the neural network solutions for a varying step size limit we see that the shape of the fit is highly dependent on this parameter. In figure \ref{fig:step size} the number of nodes and epochs was 1000 and 100 respectively. We used the $\tanh$ activation function in this case. Even though a high number of nodes and epochs allows for a better result over all, this can only partially cover up for a poor step size as the total amount of calculations increases and the program can end up being to costly from a computational point of view. The two plots above indicate the importance of having at least a decent step size so that reasonable calculational efficiency can be achieved.

	 \begin{figure}[H]
	\centering
	\begin{subfigure}{0.5\textwidth}
		\centering
		\includegraphics[width=\linewidth]{result/Resultater_supervised/arctan_1000_100_0001.pdf}
		\caption{100 epochs for the arctan activation function}
	\end{subfigure}%
	~ 
	\begin{subfigure}{0.5\textwidth}
		\centering
		\includegraphics[width=\linewidth]{result/Resultater_supervised/arctan_1000_1000_0001.pdf}
		\caption{1000 epochs for the arctan activation function}
	\end{subfigure}
	\caption{Neural network results compared to the RK4 solution for two different number of epochs}
	\label{fig:epochs}
	\end{figure}

	In figure \ref{fig:epochs} we can see how the number of epochs in the neural network changes the accuracy of the results. When the number of epochs is too low we generally get much worse initial values for all of the populations S, I and R. In this case the arctan activation function was used with 1000 nodes and 0.0001 as step size. The difference is one order of magnitude in the number of epochs (from 100 to 1000), and the results improve significantly. Note that the shape in this case is still quite poor as the step size is quite low. 


	\begin{figure}[H]
	\centering
	\begin{subfigure}{0.5\textwidth}
		\centering
		\includegraphics[width=\linewidth]{result/Resultater_supervised/arctan_100_1000_0001.pdf}
		\caption{Using arctan as the activation function}
	\end{subfigure}%
	~ 
	\begin{subfigure}{0.5\textwidth}
		\centering
		\includegraphics[width=\linewidth]{result/Resultater_supervised/swish_100_1000_0001.pdf}
		\caption{Using swish as the activation function}
	\end{subfigure}
	%\label{fig:tc-E-M}
	\begin{subfigure}{0.5\textwidth}
		\centering
		\includegraphics[width=\linewidth]{result/Resultater_supervised/tanh_100_1000_0001.pdf}
		\caption{Using tanh as the activation function}
	\end{subfigure}
	\caption{Neural network solution for three different activation functions}
	\end{figure}
	At lower values for number of nodes, epochs and step sizes the differences in the activation functions are more prominent. The neural network was run with 100 nodes, 1000 epochs and 0.0001 step size here. The swish function, although the most accurate when allowed for more computationally intensive jobs, does not converge correctly here and adopts a more linear nature in this domain. It is therefore only valid when using more epochs and nodes and with a bigger step size.
	
	\begin{figure}[H]
		\centering
		\includegraphics[width=0.5\linewidth]{result/Resultater_supervised/swish_1000_10000_01}
		\caption{Neural network solution using the swish activation function, 1000 nodes, 10000 epochs and 0.1 step size limit}
		\label{fig:swish}
	\end{figure}
	The best results was achieved with the swish function with a high number of nodes, epochs and a big step size (1000, 10000 and 0.1 respectively). In this case the results are very consistent with the RK4 method as can be seen in figure \ref{fig:swish} where the data coincide almost completely. 
	
	For many, many more figures and a full overview of the results for both supervised and unsupervised learning and the R2-score and MSE for each run, see; \href{https://github.com/sondrt/FYS-STK4155/tree/master/Project3/TEX/result}{github}.
 





\subsubsection{Monte Carlo simulation of simple SIRS model, using different time.}
	Firstly we quickly understood that one year was simply not enough time to get to a proper stady-state, so we look at how the simulation would look if we ran it for $1,10,100,1000$ years. \ref{fig:MCb1Tchanges} And we quickly figured that $5$ to $10$ years should suffice, depending on what effects we want to observe. Here the both the equilibrium line, same as before, and the mean of the Monte Carlo simulation are plottet, and even after $1000$ years, these lines were still off by $1$ or $2$. 

\begin{figure}[H]
    \centering
    \begin{subfigure}{0.49\textwidth}
        \centering
        \includegraphics[width=\linewidth]{../fig/texfig/MC_T1.png}
        \caption{$T = 1$}
    \end{subfigure}%
     ~ 
    \begin{subfigure}{0.49\textwidth}
         \centering
         \includegraphics[width=\linewidth]{../fig/texfig/MC_T10.png}
         \caption{$T = 10$}
    \end{subfigure}
     ~ 
    \begin{subfigure}{0.49\textwidth}
         \centering
         \includegraphics[width=\linewidth]{../fig/texfig/MC_T100.png}
         \caption{$T = 100$}
    \end{subfigure}
     ~ 
    \begin{subfigure}{0.49\textwidth}
         \centering
         \includegraphics[width=\linewidth]{../fig/texfig/MC_T1000.png}
         \caption{$T = 1000$}
    \end{subfigure}
    \caption{Monte Carlo simulation of the simple SIRS model, the dotted lines are the equilibrium values and the dot-dashed lines are the mean of the functions. As we can see from the plot they vary }
    \label{fig:MCb1Tchanges}
\end{figure}

\subsubsection{Monte Carlo simulation of simple SIRS model, testing recovery rate}
	In these figures we notice something spectacular! The program crashes at higher recovery rate! This is due to the fact that there are no more infected people at some point in the program, and therefor no reason to track the infections development, if there is no infection. There are theoretical steady - state for the recovery rate,$b$,- values that are smaller then the transmission rate ,$a$, -values, but due to the randomness of the Monte Carlo Simulations the infection tend to die out by itself, which makes a lot of sense in a population that fluctuate. From here on out we will use $b=1$ in all figures. 
	 
\begin{figure}[H]
    \centering
    \begin{subfigure}{0.49\textwidth}
        \centering
        \includegraphics[width=\linewidth]{../fig/texfig/MC_b1T3.png}
        \caption{$T = 3$ and $b =1$}
    \end{subfigure}%
     ~ 
    \begin{subfigure}{0.49\textwidth}
         \centering
         \includegraphics[width=\linewidth]{../fig/texfig/MC_b2T3.png}
         \caption{$T = 3$ and $b =2$}
    \end{subfigure}
     ~ 
    \begin{subfigure}{0.49\textwidth}
         \centering
         \includegraphics[width=\linewidth]{../fig/texfig/MC_b3T3.png}
         \caption{$T = 3$ and $b =3$}
    \end{subfigure}
     ~ 
    \begin{subfigure}{0.49\textwidth}
         \centering
         \includegraphics[width=\linewidth]{../fig/texfig/MC_b4T3.png}
         \caption{$T = 3$ and $b =4$ }
    \end{subfigure}
         ~ 
    \begin{subfigure}{0.6\textwidth}
         \centering
         \includegraphics[width=\linewidth]{../fig/newfig/MCRK4_b1T3.pdf}
         \caption{$T = 3$ and $b =1$, both MC and RK4 plottet on top of each other.}
    \end{subfigure}
    \caption{}
    \label{fig:MCSIRS}
\end{figure}

\subsection{SIRS modelling with Improvements}
Here we present the results form the different versions of SIRS, Vital dynamics, Seasonal variation and Vaccination, and some of these combined. 

\subsubsection{Vital dynamics}
Vital dynamics are here included as presented in the theory section \ref{sec:VD} for both RK4 and MC. There is some difference in the given death/birth - rates between RK4 and MC due to lack of normalization $dt$. It is easy to see that these numbers are vital for both the survival and death of the disease

\begin{figure}[H]
    \centering
    \begin{subfigure}{0.49\textwidth}
        \centering
        \includegraphics[width=\linewidth]{../fig/newfig/RK4_vitalDynamics_d001_dI01_e05.pdf}
        \caption{High birth-rate($e=0.05$), low death-rate, both in general, but also for the infected. ($d = 0.001, d_I =0.01$)}
    \end{subfigure}%
     ~ 
    \begin{subfigure}{0.49\textwidth}
         \centering
         \includegraphics[width=\linewidth]{../fig/newfig/RK4_vitalDynamics_d001_dI01_e005.pdf}
         \caption{low birth-rate ($e = 0.005$), low death-rate, both in general, but also for the infected ($d = 0.001, d_I =0.01$)}
    \end{subfigure}
     ~ 
    \begin{subfigure}{0.49\textwidth}
         \centering
         \includegraphics[width=\linewidth]{../fig/newfig/RK4_vitalDynamics_d01_dI30_e05.pdf}
         \caption{high birth-rate ($e=0.05$), high general death-rate and very high death-rate for the infected.}
    \end{subfigure}
     ~ 
    \begin{subfigure}{0.49\textwidth}
         \centering
         \includegraphics[width=\linewidth]{../fig/newfig/RK4_vitalDynamics_d002_dI10_e006.pdf}
         \caption{low birth-rate ($e=0.006$), low general death-rate($0.002$), high death-rate for infected($d_I = 0.10$)}
    \end{subfigure}
    \caption{Looking at different birth-, death-, infected death - rates.}
    \label{fig:RK4VD1}
\end{figure}


\begin{figure}[H]
    \centering
    \begin{subfigure}{0.49\textwidth}
        \centering
        \includegraphics[width=\linewidth]{../fig/texfig/MC_vitaldynamic_d0002_dI0004_e0006.png}
        \caption{$d = 0.0002, d_I = 0.0004, e = 0.0006$}
    \end{subfigure}
     ~ 
    \begin{subfigure}{0.49\textwidth}
         \centering
         \includegraphics[width=\linewidth]{../fig/texfig/MC_vitaldynamic_d00002_dI0001_e00006.png}
         \caption{$d = 0.0002, d_I = 0.0001, e = 0.0006$}
    \end{subfigure}
     ~ 
    \begin{subfigure}{0.49\textwidth}
         \centering
         \includegraphics[width=\linewidth]{../fig/texfig/MC_vitaldynamic_d00002_dI001_e00006.png}
         \caption{$d = 0.0002, d_I = 0.001, e = 0.0006$}
    \end{subfigure}
     ~ 
    \begin{subfigure}{0.49\textwidth}
         \centering
         \includegraphics[width=\linewidth]{../fig/texfig/MC_vitaldynamic_d00002_dI003_e00006.png}
         \caption{$d = 0.0002, d_I = 0.03, e = 0.0006$}
    \end{subfigure}
    \caption{Main difference between the MC method and the RK4 is that $dt$ was not normalized in the cod, so smaller numbers had a much bigger impact.}
    \label{fig:RK4VD2}
\end{figure}


\pagebreak
\subsubsection{Seasonal Variation}
Here a Seasonal Variation is included \ref{sec:SV}, The value $A$ represents the maximum deviation from the infection rate $a_0=4$.  
\begin{figure}[H]
    \centering
    \begin{subfigure}{0.49\textwidth}
        \centering
        \includegraphics[width=\linewidth]{../fig/newfig/Vaccine_A=12_T=5.pdf}
        \caption{RK4 $A=1.2$}
    \end{subfigure}%
     ~ 
    \begin{subfigure}{0.49\textwidth}
         \centering
         \includegraphics[width=\linewidth]{../fig/newfig/Vaccine_A=20_T=5.pdf}
         \caption{RK4$A=2.0$}
    \end{subfigure}
     ~ 
    \begin{subfigure}{0.49\textwidth}
         \centering
         \includegraphics[width=\linewidth]{../fig/texfig/MCSV_A=12T=5.png}
         \caption{MC $A=12$ }
    \end{subfigure}
     ~ 
    \begin{subfigure}{0.49\textwidth}
         \centering
         \includegraphics[width=\linewidth]{../fig/texfig/MCSV_A=20T=5.png}
         \caption{MC $A = 20$ }
    \end{subfigure}
    \caption{This figure shows the impact of seasonal variations on RK4 and MCS, notice how the seasonal variation impact is delayed in the MC scheme.}
    \label{fig:RK4VD3}
\end{figure}




\pagebreak
\subsubsection{Vaccination}
	If we assume that vaccines work and the earth is round, than we can see what incredible tool vaccines are for managing diseases. These results shows how efficient vaccine in combination with seasonal variation can be, when you only vaccinate $10\%$ of the population once per year.
\begin{figure}[H]
    \centering
    \begin{subfigure}{0.30\textwidth}
        \centering
        \includegraphics[width=\linewidth]{../fig/newfig/MC_onlyvaccineT=5f=40.pdf}
        \caption{MC vaccine}
    \end{subfigure}
     ~ 
    \begin{subfigure}{0.30\textwidth}
         \centering
         \includegraphics[width=\linewidth]{../fig/newfig/MC_vaccine_SVA=4_T=5f=40_notiming.pdf}
         \caption{MC vaccine no timing}
    \end{subfigure}
     ~ 
    \begin{subfigure}{0.30\textwidth}
         \centering
         \includegraphics[width=\linewidth]{../fig/newfig/MC_vaccine_SVA=4_T=5f=40_timing.pdf}
         \caption{RK4 vaccine seasonal timing}
    \end{subfigure}
     ~ 
    \begin{subfigure}{0.30\textwidth}
         \centering
         \includegraphics[width=\linewidth]{../fig/newfig/RK4Vaccine_f=40_T=5.pdf}
         \caption{RK4 vaccine}
    \end{subfigure}
     ~ 
    \begin{subfigure}{0.30\textwidth}
         \centering
         \includegraphics[width=\linewidth]{../fig/newfig/RK4Vaccine_SVA=2_f=40_T=5.pdf}
         \caption{RK4 vaccine no  timing}
    \end{subfigure}
     ~ 
    \begin{subfigure}{0.30\textwidth}
         \centering
         \includegraphics[width=\linewidth]{../fig/newfig/RK4Vaccine_SVA=4_f=40_T=5_timing.pdf}
         \caption{RK4 vaccine seasonal timing}
    \end{subfigure}
    \caption{The SIRS model, figure a) and d) are clean SIRS models with the effect of a yearly vaccine, b) and e) include the seasonal variation of the disease, lastly in figure c) and f) we have timed the vaccine to be given right before the winter, where it is assumed to be at its worst. (at least in this model).}
    \label{fig:MCVaccination}
\end{figure}




