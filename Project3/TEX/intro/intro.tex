
Systems dynamics models are representations of the real world made by dividing the population up into categories, with accompanying mathematical representations of how these categories interact with each other and how members of one category move to another. These models have a long history of use in epidemiology, appearing in rudimentary form in work by Bernoulli, for example. Due to their history and mathematical tractability these models are fundamental tools in the modeling of human health. Many other type of models, as agent-based or network models do exist but they will not be addressed here. 

	Here we will address how to construct a system dynamics model of the classical SIRS model of epidemiology, and how to solve the model computationally with a 4th order the Runge-Kutta method, and a deterministic approach with the Monte Carlo method. If this model where a success at predicting the flue, influenza, then that would provide public health officials valuable advanced warnings that could come aid in efforts to reduce this burden.\cite{yang2014comparison} And finally explore some more complex examples of the SIRS model where I include vital dynamics, Seasonal Variation and Vaccination	
	The main purpose of creating such a simulation is to investigate how a disease spreads throughout a given population over time. So that we can make predictions on wether or not a disease has the capability to establish itself in a population. 