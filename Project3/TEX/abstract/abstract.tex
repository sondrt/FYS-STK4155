%Abstract


The goal of this project was to develop a Monte Carlo simulation of the spread of an infectious disease. This was done with the classical SIRS model of epidemiology, and it was studied and used to develop the necessary transition probabilities for the simulation. 

A deterministic approach, the  4th order Runge-Kutta,  was used in conjunction with the Monte Carlo method on the classical SIRS model. These where then tested and improved to suffice for different scenarios and diseases. 

We found that the Monte Carlo is a solid and semi-realistic approach to fluctuations in diseases, given enough data.

This is far form a done analyzes; The model should be tested on real data, parameters should be added; incubation time could be one,  and the code should go through a major makeover so that it is easier to expand the model and more intuitive to use. It should also be made a simple way of tracing individuals in the group.