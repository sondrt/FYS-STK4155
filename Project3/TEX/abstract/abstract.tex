%Abstract

This report deals with the numerical solution of the coupled ordinary differential equations provided by the classical SIRS model of epidemiology. The traditional 4th order Runge-kutta(RK4) scheme is compared with the novel approach of using neural networks(NN), and Monte Carlo simulations(MCs). Autograd is used for the NN due to its ability to handle coupled ODEs. 
	
	While RK4 is better both in terms of accuracy and runtime, the NN also showed results, giving a mean squared on the order $10^{-6}$ after $10000$ epochs.The swish method was determined to be the best activation function. The MCs were a solid and closer to a real scenario of the disease. Of this reason the report proceeds with improving the SIRS model for the RK4 solution and MCs by including; Vital dynamics, Seasonal Variation, and Vaccination. Where it was again showed that vaccines, with propper management, are a crucial tool to hinder the potential spread of dangerous diseases.