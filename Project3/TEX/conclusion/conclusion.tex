

The partial ordinary differential equations,SIRS, were solved with both RK4, MCs, and a neural networks. While the NN converged towards the analytical solution, they were clearly beaten by RK4, which performed better in terms of both accuracy and runtime. The main conclusion is, as in the previous project, to use specialised methods whenever possible, while neural networks can offer an acceptable solution which does not require much knowledge from the user, especially when using a library like Autograd.

A parameter sweep was performed in order to find the optimal solution of activation functions, optimisers and network architectures. Few layers with more nodes performed best. The network performed best when using the swish function as a activation function. Furthermore, the choice of transformation of the output from the neural network proved important for good performance, and the choice that used the most information about the solution proved most effective, giving a mean squared deviation on the order of $10^{-6}$ after $10000$ epochs.


It seems that the SIRS model is a great system dynamics model for representing real diseases, in general a great foundation to build on for predicting disease , and how they establish themselves within the population. We have also seen that it is very versatile, with a few improvements tested out, and combined in this report. It was shown that a Monte Carlo simulation were a great way of approaching the differential equations in a more realistic manner, if given a large enough dataset, Neural network can for sure have its uses, but at the moment RK4 outperforms with this setup. 